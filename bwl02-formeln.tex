\documentclass[a4paper,12pt]{scrartcl} 

%\usepackage[latin1]{inputenc} 
%Apple \usepackage[applemac]{inputenc} 
\usepackage[utf8]{inputenc}
\usepackage[ngerman]{babel}
\usepackage[T1]{fontenc}

%Das Paket erzeugt ein anklickbares Verzeichnis in der PDF-Datei.
\usepackage[hyperfootnotes=false,colorlinks=true,linkcolor=black,urlcolor=black]{hyperref}

%Das Paket wird fr die anderthalb-zeiligen Zeilenabstand bentigt
\usepackage{setspace}

%Einr�ckung eines neuen Absatzes
\setlength{\parindent}{0em}

%Definition der Rnder
\usepackage[paper=a4paper,left=30mm,right=30mm,top=30mm,bottom=30mm]{geometry} 

\usepackage{amsfonts}
\usepackage{amsmath}
\usepackage{cancel}
\usepackage{graphicx}
\usepackage{mathcomp}
\usepackage{polynom}


%Abstand der Fu�noten
\deffootnote{1em}{1em}{\textsuperscript{\thefootnotemark\ }}

%Regeln, bis zu welcher Tiefe (section,subsection,subsubsection) berschriften angezeigt werden sollen (Anzeige der berschriften im Verzeichnis / Anzeige der Nummerierung)
\setcounter{tocdepth}{3}
\setcounter{secnumdepth}{3}

%-------------------
%Ende des Kopfbereiches
%-------------------




%-------------------
%Hier beginnt der Text deiner Hausarbeit
%-------------------
\begin{document}


%Beginn der Titelseite
\begin{titlepage}
\begin{small}
\vfill {AKAD\\ 
Bachelor of Science (Wirtschaftsinformatik) \\ 
Modulzusammenfassung}
\end{small}


\begin{center}
\begin{Large}
\vfill {\textsf{\textbf{
Bwl02 \\
\vspace*{1cm} 
Formelsammlung
}}}
\end{Large}
\end{center}

\begin{small}
\vfill Daniel Falkner \\ Rotbach 529 \\  94078 Freyung \\  daniel.falkner@akad.de \\ 
\today
\end{small}

\end{titlepage}
%Ende der Titelseite


%Inhaltsverzeichnis (aktualisiert sich erst nach dem zweiten Setzen)
\tableofcontents
\thispagestyle{empty}

%Beginn einer neuen Seite
\clearpage

%Anderthalbzeiliger Zeilenabstand ab hier
\onehalfspacing

\pagestyle{plain}


\section{Formeln}

\subsection{Eigenkapital}
Eigenkapital = Aktiva - Passiva
\subsection{Gesamtkosten}
Gesamtkosten = Fixkosten + (Menge * variable Stückkosten)
\subsection{Erfolg}
Erfolg = Ertrag - Aufwand
\subsubsection{Erfolg je Kostenträger}
Erfolg je Kostenträger = Erlöse - Selbstkosten 
\subsection{Erlös}
Erlös = Menge * Verkaufspreis
\subsection{Deckungsbeitrag}
Deckungsbeitrag = Erlös - variable Kosten
\subsection{BEP} 
BEP \footnote{Break Even Point} = $\cfrac{Fix~Kosten}{St"uckpreis - variable~ Kosten}$ 
\\
\\
BEP  = $\cfrac{Fix~Kosten}{Deckungsbeitrag}$ 


\subsection{Marktanteil}
Marktanteil = $\cfrac{Unternehmensumsatz}{Branchenumsatz}$
\subsubsection{Relativer Marktanteil}
Relativer Marktanteil = $\cfrac{Unternehmensumsatz}{Umsatz~st"arkster~Konkurrent}$

\subsection{Leistung je Mitarbeiter}
Leistung je Mitarbeiter = $\cfrac{Umsatz}{Mitarbeiteranzahl}$
\subsection{Werbeaufwand am Umsatz}
Anteil Werbeaufwand am Umsatz = $\cfrac{Werbeaufwand~Produkt~X}{Umsatz~Produkt~X}$


%% Kennzahlen des Produktionsmanagement 

\subsection{Kapzität}
Kapazität = $\cfrac{Menge~des~Outputs}{Zeiteinheit}$


\subsection{Beschäftigungsgrad}
Beschäftigungsgrad = $\cfrac{tatsächliche~Leistung}{Maximalleistung~(Kapazit"at)}$

\subsection{Produktivität}
Produktivität = $\cfrac{Ausbringungsmenge}{Einsatzmenge}$ = $\frac{Output}{Input}$ 
\subsubsection{Arbeits Produktivität}
Arbeits Produktivität = $\cfrac{Produktionsmenge}{Arbeiter/Arbeitsstunden}$
\subsubsection{Maschinen Produktivität}
Maschinen Produktivität = $\cfrac{Produktionsmenge}{Maschinen/Maschinensstunden}$

\subsection{Wirtschaftlichkeit}
Wirtschaftlichkeit = $\cfrac{Ausbringungswert}{Kosten~der~Produktionsfaktoren} = \cfrac{Leistung}{Kosten}$ 

\subsection{Rentabilität} 
Rentabilität = $\cfrac{Gewinn}{Kapital} * 100$ 

\subsection{ROI} 
ROI \footnote{Return-On-Invest} = $\cfrac{Gewinn}{Umsatz} * \cfrac{Umsatz}{Kapital} * 100$ 

\subsection{Servicegrad} 
Servicegrad = $\cfrac{Anzahl~befriedigter~Bedarfsanforderungen}{Anzahl~Bedarfsanforderungen}$ 


\subsection{Umschlagshäufigkeit}
Umschlagshäufigkeit = $\cfrac{Verbauch~pro~Periode}{Durchschnittlicher~Lagerbestand}$
\subsubsection{Kapitalumschlagshäufigkeit}
Kapitalumschlagshäufigkeit = $\cfrac{Umsatz}{Kapital}$

\subsection{Durchschnittliche Lagerdauer} 
Durchschnittliche Lagerdauer = $\cfrac{360}{Umschlagsh"aufigkeit}$ 

\subsection{Umsatzrendite} 
Umsatzrendite = $\cfrac{Gewinn}{Umsatz} * 100$ 





\end{document}

